\documentclass[avery5371,fronts,grid,frame,a4paper]{flashcards}
\usepackage[utf8]{inputenc}
\usepackage[german]{babel}
%\newcommand{\cardpaper}{a4paper}
%\newcommand{\cardpapermode}{portrait}
\renewcommand{\cardrows}{4}
\renewcommand{\cardcolumns}{2}
\cardfrontstyle[\large\slshape]{headings}
\cardbackstyle{empty}

\newcommand\question[2]{  \begin{flashcard}[{\chap} -- #1]{#2}
%    #2
  \end{flashcard}
}
\newcommand\class[1]{{\footnotesize [Klassen: #1]}}

\begin{document}
% Stand 15. November 2014
% http://www.bmvit.gv.at/bmvit/telekommunikation/funk/funkdienste/downloads/amateur_fragen.pdf

% Rechtliches

\def\chap{Rechtliches \class{1,3,4}}

\question{01}{Welche gesetzlichen Bestimmungen sind für den Amateurfunk maßgeblich?}
\question{02}{Was ist die ITU?}
\question{03}{Welche Zwecke verfolgt der internationale Fernmeldevertrag?}
\question{04}{Welche Aufgaben hat das Radiocommunication Bureau? Was ist die CEPT und welche Bedeutung hat sie?}
\question{06}{Was ist die VO Funk (Radio Regulations) und was regelt sie?}
\question{07}{Definieren Sie den Begriff ,,Funkanlage'' im Sinne des TKG.}
\question{08}{Erläutern Sie den Unterschied zwischen einem Telekommunikationsdienst und dem Amateurfunkdienst?}
\question{09}{Wann erlischt eine Bewilligung? Was kann passieren, wenn Sie ohne oder ohne entsprechende Amateurfunkbewilligung Amateurfunk betreiben?}
\question{11}{Welche Funkanlagen sind bewilligungspflichtig, welche Art der Bewilligungen gibt es?}
\question{12}{Sie ändern den Standort Ihrer Funkanlage – was haben Sie zu tun?}
\question{13}{Was versteht man unter dem Aufsichtsrecht der Fernmeldebehörden über Telekommunikationsanlagen?}
\question{14}{Ein Organ der Fernmeldebehörde will ihre Funkanlage überprüfen, was haben Sie zu tun? Welche Geheimhaltungspflichten treffen Sie als Funkamateur?}
\question{16}{Was kann die Fernmeldebehörde machen, falls Sie einen anderen Funkdienst stören?}
\question{17}{Welche Gebühren müssen als Funkamateur entrichtet werden?}
\question{18}{Definieren Sie den Begriff ,,Amateurfunkdienst''?}
\question{19}{Definieren Sie den Begriff ,,Funkamateure''? Definieren Sie den Begriff ,,Amateurfunkstelle''?}
\question{21}{Definieren Sie den Begriff ,,Stationsverantwortlicher''?}
\question{22}{Definieren Sie den Begriff ,,Klubfunkstelle''?}
\question{23}{Definieren Sie den Begriff ,,Bakensender''?}
\question{24}{Definieren Sie den Begriff ,,Relaisfunkstelle''? Darf Amateurfunk von Nichtamateuren abgehört werden?}
\question{26}{Voraussetzungen zur Erlangung einer Amateurfunkbewilligung?}
\question{27}{Wie und wo ist ein Antrag auf Erteilung einer Amateurfunkbewilligung zu stellen?}
\question{28}{Rufzeichen und Sonderrufzeichen?}
\question{29}{Wozu berechtigt eine Amateurfunkbewilligung?}
\question{30}{Unter welchen Voraussetzungen dürfen Aussendungen durchgeführt werden?}
\question{31}{Wie ist der Amateurfunkverkehr abzuwickeln?}
\question{32}{Definieren Sie den Begriff Not- und Katastrophenfunkverkehr?}
\question{33}{Wo können Sie erfahren, unter welchen technischen Parametern (Sendeart, Leistungsstufe, Einschränkungen, etc.) Sie mit Ihrer Lizenzklasse in welchem Frequenzband Amateurfunk betreiben dürfen?}
\question{34}{Was ist ein und wozu gibt es ein Funktagebuch?}
\question{35}{In welchem Umfang ist Mitbenutzung einer Amateurfunkstelle möglich?}
\question{36}{Wer ist für Amtshandlungen nach dem Amateurfunkgesetz zuständig?}
\question{37}{Nennen Sie einige Verwaltungsstrafbestimmungen in Bezug auf den Amateurfunk?}
\question{38}{Was ist eine CEPT-Lizenz? \\ (oder CEPT-Novizen-Lizenz)}
\question{39}{Was darf ein ausländischer CEPT-Lizenz Inhaber oder CEPT-Novizen-Lizenz in Österreich ohne eigene österreichische Bewilligung?}
\question{40}{Was bedeutet der Begriff Reziprozität und nennen Sie ein Beispiel?}
\question{41}{Nennen Sie die Bewilligungsklassen und wozu berechtigen diese?}
\question{42}{Welche Leistungsstufen kennen Sie und nennen Sie deren Merkmale?}
\question{43}{Unter welchen Voraussetzungen kann eine Amateurfunkbewilligung für die Leistungsstufe C erteilt werden?}
\question{44}{Unter welchen Voraussetzungen kann eine Amateurfunkbewilligung für die Leistungsstufe D erteilt werden?}
\question{45}{Was bedeutet der Status eines Funkdienstes (Primär, Primär/Exklusiv(Pex), Sekundär, ISM)?}
\question{46}{Ist die Verwendung der Betriebsart Telegraphie an eine bestimmte Voraussetzungen gebunden?}
\question{47}{Wann wird eine schädliche Störung als solche behandelt?}
\question{48}{Was gilt für einen Amateurfunkbetrieb auf Schiffen und in Flugzeugen?}
\question{49}{Welche Aussendungen dürfen von einer Amateurfunkstelle empfangen werden?}
\question{50}{Was darf der Nachrichteninhalt einer Amateurfunkaussendung sein?}
\question{51}{Gibt es eine Möglichkeit, dass ein Funkamateur, der die Prüfungskategorie 3 erfolgreich abgelegt hat, auf anderen Frequenzen als dem 2m / 70cm-Band Funkverkehr haben darf?}
\question{52}{Wer darf eine Relaisfunkstelle errichten / betreiben / benutzen und wie ist deren Rufzeichen auszusenden?}
\question{53}{Was haben Sie zu tun, wenn Sie Funkverkehr mit einer nicht bewilligten Amateurfunkstelle haben und mit wem dürfen Sie keinen Amateurfunkverkehr haben?}
\question{54}{Welche besonderen Aufgaben hat die ITU in Bezug auf Funkdienste und welche Ausschüsse sind dafür zuständig?}
\question{55}{Was bedeutet missbräuchliche Verwendung von Funkanlagen?}
\question{56}{Was hat der Inhaber einer Amateurfunkstelle zu tun, wenn er nicht bei dieser Stelle anwesend ist?}
\question{57}{Welche Bestimmungen sind beim Betrieb einer Amateurfunkstelle im Ausland zu beachten?}
\question{58}{Unter welchen Voraussetzungen darf der Inhaber einer Amateurfunkbewilligung der Bewilligungsklasse 3 im Ausland Amateurfunkbetrieb durchführen?}
\question{59}{Wozu berechtigt eine Amateurfunkbewilligung der Klasse 4?}
\question{60}{Aufgrund welcher internationalen Regelung dürfen Funkamateure aus bestimmten Ländern auch ohne individuelle Gastzulassung vorübergehend in Österreich Amateurfunk ausüben?}
\question{61}{Unter welchen Voraussetzungen ist die Verbindung von Amateurfunkstellen mittels Internettechnologie zulässig?}


\def\chap{Betrieb und Fertigkeiten \class{1,4}}

\question{01}{Wie eröffnen Sie einen Funkverkehr in Phonie, wie in Telegraphie?}
\question{02}{Was ist das gebräuchliche Minimum einer Amateurfunkverbindung?}
\question{03}{Welche Bedeutung haben die Q-Gruppen im allgemeinen?
  \begin{center}
    QRM \quad QSO \quad QSY \quad QSL \quad QRP \quad QTR
  \end{center}
}
\question{03}{Welche Bedeutung haben die Q-Gruppen im allgemeinen?
  \begin{center}
    QRS \quad QRX \quad QRO \quad QRV \quad QSP \quad QRG
  \end{center}
}
\question{03}{Welche Bedeutung haben die Q-Gruppen im allgemeinen?
  \begin{center}
    QRT \quad QRU \quad QRN \quad QRB \quad QTH \quad QSB
  \end{center}
}

%\begin{itemize}
%  \item QRM ich werde gestört (Fremdstörungen),
%  \item QSO ich habe Verbindung mit \dots (im Amateurgebrauch auch Bezeichnung für eine Funkverbindung)
%  \item QSY wechseln Sie auf die Frequenz \dots kHz (im Amateurgebrauch statt einer Frequenz oft das Amateurband),
%  \item QSL ich werde Ihnen eine Empfangsbestätigung geben (im Amateurgebrauch allgemeiner Hinweis, dass eine Meldung verstanden wurde u n d Bezeichnung für die ,,Funkbestätigungskarte/QSL-Karte'')
%  \item QRP vermindern Sie die Sendeleistung (im Amateurgebrauch auch Hinweis, dass mit geringer Sendeleistung gearbeitet wird)
%  \item QTR es ist \dots Uhr GMT (UTC)
%  \item QRS geben Sie langsamer (eventuell gefolgt von der erwünschten Anzahl Worte pro Minute =WPM),
%  \item QRX ich werde Sie um \dots Uhr auf \dots kHz wieder rufen (im Amateurgebrauch als allgemeiner Hinweis, dass man später gerufen wird, derzeit aber warten soll),
%  \item QRO erhöhen Sie Ihre Sendeleistung
%  \item QRV ich bin betriebsbereit
%  \item QSP ich werde an \dots weiterübermitteln,
%  \item QRG Ihre genaue Frequenz ist \dots kHz
%  \item QRT stellen Sie die Aussendung(en) ein (im Amateurgebrauch auch für ,, ich stelle den Funkbetrieb ein!'')
%  \item QRU ich habe nichts für Sie vorliegen (im Amateurgebrauch die Mitteilung, dass alle Informationen übermittelt wurden; wird am Ende eines QSOs verwendet),
%  \item QRN ich habe atmosphärische Störungen (1 = keine, 2 = schwach, 3 = mäßige, 4 = starke, 5 = sehr starke),
%  \item QRB die Entfernung zwischen unseren beiden Stationen ist....km
%  \item QTH mein Standort ist \dots
%  \item QSB Ihre Zeichen weisen Fading auf (= die Empfangsfeldstärke schwankt).
%\end{itemize}

\question{04}{Sie wollen, dass Ihre Gegenstation die Sendeleistung vermindert. Welche Q-Gruppe verwenden Sie?}
\question{05}{Was bedeuten die Hinweise \\ ,,5 UP'' bzw. ,,10 DOWN''?}
\question{06}{Sie wollen in einen bestehenden Funkverkehr einsteigen. Wie führen Sie das durch?}
\question{07}{Welche betrieblichen Auswirkungen haben die besonderen Ausbreitungsbedingungen auf Kurzwelle?}
\question{08}{Welche betriebliche Auswirkung hat die Bodenwellen-Ausbreitung?}
\question{09}{Welche betriebliche Auswirkung hat die Raumwellen-Ausbreitung, in welchem Frequenzbereich ist sie von Bedeutung?}
\question{10}{Welche betriebliche Bedeutung hat die kritische Frequenz?}
\question{11}{Welche betriebliche Bedeutung haben die Begriffe ,,MUF'' und ,,LUF''?}
\question{12}{Was versteht man unter Fading auf Kurzwelle, wodurch entsteht Fading und wie reagieren Sie, um den Funkverkehr aufrecht zu erhalten?}
\question{13}{Ausbreitung von Funkwellen -- Ausbreitungsmerkmale in den verschiedenen Amateurfunk Frequenzbereichen?}
\question{14}{Welchen Einfluß hat die Ionosphäre auf die Ausbreitung von Funkwellen über 30 MHz?}
\question{15}{Erklären Sie die Begriffe Fresnelzone, Geländeschnitt}
\question{16}{Was ist die tote Zone? Was ist ein Skip?}
\question{17}{Wovon hängt die maximal erzielbare Reichweite auf Kurzwelle ab?}
\question{18}{Was verstehen Sie unter kurzem Weg? Was unter langem Weg?}
\question{19}{Was verstehen Sie unter dem Dämmerungseffekt?}
\question{20}{Was verstehen Sie unter der ,,Grey-Line'', welche Besonderheiten in der Funkausbreitung können auftreten?}
\question{21}{Beschreiben Sie den Aufbau der lonosphäre und welche betriebliche Konsequenzen ergeben sich daraus?}
\question{22}{Wie verhalten sich die Ionosphärenschichten im Tagesverlauf bzw. im Jahresverlauf?}
\question{23}{Welchen Einfluss hat die geographische Breite auf die Kurzwellenausbreitung?}
\question{24}{Was versteht man unter Sonnenaktivität, unter der Sonnenfleckenrelativzahl, unter dem ,,Solar-Flux''? Welchen Einfluss hat sie auf die Kurzwellenausbreitung?}
\question{25}{Welchen Zyklen unterliegen die Ausbreitungsbedingungen auf Kurzwelle?}
\question{26}{Beschreiben Sie das charakteristische Ausbreitungsverhalten in den dem Amateurfunkdienst zugewiesenen Frequenzbändern unter 30 MHz?}
\question{27}{Was versteht man unter einem Mögel-Dellinger-Effekt und welche betriebliche Auswirkungen hat er?}
\question{28}{Welche Auswirkungen haben Polarlicht-Erscheinungen auf die Kurzwellenausbreitung?}
\question{29}{Welche Faktoren können den Funkbetrieb auf Kurzwelle beeinflussen?}
\question{30}{Wie wirkt sich die Tageszeit auf die Ausbreitung in den Kurzwellenbändern bis 40m aus? (160m/80m-/40m-Band)}
\question{31}{Was verstehen Sie unter ,,Sporadic E-Verbindungen''?}
\question{32}{Was verstehen Sie unter ,,Short-Skips''?}
\question{33}{Was verstehen Sie unter einem Notverkehr, wie wird er angekündigt?}
\question{34}{Sie empfangen einen Notruf – woran erkennen Sie diesen und wie haben Sie sich zu verhalten?}
\question{35}{Auf welchen Bändern könnten Sie einen Notruf empfangen?}
\question{36}{Welche Sendearten sind im Kurzwellenbereich zulässig?}
\question{37}{Müssen Sie ein Funktagebuch führen und welche Angaben muss es enthalten?}
\question{38}{Was verstehen Sie im Telegraphiebetrieb unter ,,BK-Verkehr''?}
\question{39}{Was verstehen Sie unter UTC (GMT) -- Zusammenhang zu Lokalzeit, Sommerzeit}
\question{40}{Nennen Sie die konkreten Frequenzbereiche, die dem Amateurfunkdienst in den jeweiligen Frequenzbändern zugewiesen sind (5 Beispiele)}
\question{41}{Wie arbeiten Sie mit ausländischen Amateurfunkstationen zusammen, die einen anderen/erweiterten Bandbereich benutzen? (Beispiele: 40m, 80m)?}
\question{42}{Was bedeuten die folgenden Abkürzungen: BK, CQ, CW, DE, K?}
\question{42}{Was bedeuten die folgenden Abkürzungen: PSE, RST, R, N, UR?}
\question{42}{Was bedeuten die folgenden Abkürzungen: FB, DX, RPT, HW, CL?}
% BK engl. break (Aufforderung zur Unterbrechung)
% CQ an alle (Funkstellen)
% CW engl. continuous wave / Telegraphie
% DE von
% K kommen
% PSE engl. please / bitte
% RST Rapport (R = engl. Readability / Lesbarkeit; S = engl. Signalstrengh / Lautstärke; T = engl. Tonequality / Signalqualität, nur für CW)
% R engl. roger / verstanden
% N engl. no / nein
% UR engl your / dein, deine
% FB engl. faible / gut
% DX Weitverbindung
% RPT engl. Repeat / wiederholen
% HW engl. how? / wie?
% CL engl. close / für ,,ich schließe die Funkstelle''
\question{43}{Wie wirkt sich Polarisationsfading auf den Kurzwellenbetrieb aus?}
\question{44}{Was versteht man unter Schwund im Kurzwellenbereich und wie reagieren Sie, um den Funkverkehr aufrecht zu erhalten?}
\question{45}{Welche Maßnahmen ergreifen Sie, wenn Sie darauf aufmerksam gemacht werden, dass Ihre Aussendung ,,splattert''?}
\question{46}{Was ist ein ,,Pile-Up'' -- wie verhalten Sie sich richtig?}
\question{47}{Was verstehen Sie unter den Begriffen {\footnotesize\texttt MAYDAY - SECURITEE - SILENCE MAYDAY - MAYDAY RELAY?}}
\question{48}{Welche Mess- und Kontrollgeräte sind bei einer Amateurfunkstelle vorgeschrieben?}
\question{49}{Was ist bei der Abstimmung des Leistungsverstärkers einer Amateurfunkstelle zu beachten?}
\question{50}{Wie wird ein Funkrufzeichen allgemein bzw. ein Amateurfunkrufzeichen aufgebaut – nach welcher Vorschrift?}
\question{51}{Buchstabieren Sie folgende Worte bzw. den folgenden Text nach dem internationalen Buchstabieralphabet: \dots}
\question{52}{Was ist beim Betrieb an den Bandgrenzen zu beachten?}
\question{53}{Nennen Sie Beispiele österreichischer Amateurfunkrufzeichen mit Zusätzen (zB: am, mm, /1).}
\question{54}{Nennen Sie die Landeskenner von fünf Nachbarländern und von fünf weiteren Ländern.}
\question{55}{Was bedeuten die Ziffern im österreichischen Amateurfunkrufzeichen, welche Rufzeichenzusätze sind zulässig?}
\question{56}{Welche Bestimmungen sind beim Betrieb im 160m-Band zu beachten?}
\question{57}{Welche Betriebsverfahren werden bei Scatter-Verbindungen verwendet?}
\question{58}{Welche Betriebsverfahren werden bei Meteorscatter-Verbindungen angewendet?}
\question{59}{Erklären Sie die Betriebsabwicklung bei Relaisbetrieb.}
\question{60}{Was versteht man unter ,,EME - Verbindungen''? Welches Betriebsverfahren wird angewendet?}
\question{61}{Was verstehen Sie unter Packet Radio? Welches Betriebsverfahren wird angewendet?}
\question{62}{Was verstehen Sie unter den Begriffen Mailbox, Digipeater, Netzknoten und welche betriebliche Besonderheiten sind zu beachten?}
\question{63}{Erklären Sie die Begriffe Relaisfunkstelle, Transponder, Bakensender und welche betrieblichen Besonderheiten sind zu beachten?}
\question{64}{Erklären Sie die Betriebsabwicklung bei ATV-Betrieb.}
\question{65}{Was ist bei Überreichweitenbedingungen zu beachten?}
\question{66}{Welchen Einfluss hat die Wahl des Standortes für UKW-Ausbreitung?}
\question{67}{Erklären Sie das Betriebsverfahren SSTV.}
\question{68}{Nennen Sie Einflüsse, die die Lesbarkeit einer Funkverbindung verschlechtern.}
\question{69}{Wie beurteilen Sie die Aussendung Ihrer Gegenstelle und wie wird diese Beurteilung der Gegenstelle mitgeteilt?}
\question{70}{Wie teilen Sie der Gegenstation Ihren Standort mit?}
\question{71}{Was ist ein ,,Contest''? Wie verhalten Sie sich richtig?}
\question{72}{Wie gehen Sie bei der Planung einer Amateurfunkverbindung zu einem bestimmten Ort vor?}
\question{73}{Was ist hinsichtlich der Herstellung oder Veränderung von Amateurfunkgeräten zu beachten?}
\question{74}{Beschreiben Sie das typische Ausbreitungsverhalten in den Frequenzbändern 6m--2m und 70cm.}

\def\chap{Betrieb und Fertigkeiten \class{3}}

\question{01}{Frequenzbereich des 70cm-Amateurfunkbandes / 2m Bandes?}
\question{02}{Wie eröffnen Sie einen Sprechfunkverkehr?}
\question{03}{Wie sind Amateurfunkrufzeichen aufgebaut?}
\question{04}{Welche Zusätze zu einem Amateurfunkrufzeichen sind zulässig?}
\question{05}{Nennen Sie mindestens 5 Landeskenner der umliegenden Länder.}
\question{06}{Wie beurteilen Sie das Signal Ihrer Gegenstation?}
\question{07}{Was versteht man unter ,,S-Stufe(n)''?}
\question{08}{Was versteht man unter Not- und Katastrophenfunkverkehr, wie wird er gekennzeichnet?}
\question{09}{Wie nahe dürfen Sie beim Sendebetrieb an die Bandgrenze herangehen?}
\question{10}{Welche Sendearten sind mit der Bewilligungsklasse~3 zulässig und mit welcher maximalen Sendeleistung?}
\question{11}{Was versteht man unter einem Amateurfunkrelais, wozu dient es?}
\question{12}{Wie wickeln Sie einen Betrieb über ein Amateurfunkrelais ab?}
\question{13}{Buchstabieren Sie Ihren Vor- und Zunamen nach dem internationalen Buchstabieralphabet.}
\question{14}{Wie verhalten Sie sich beim Empfang von Signalen mit ,,Doppler - Shift''?}
\question{15}{Was versteht man unter ,,Frequenzablage'' bei Relaisbetrieb?}
\question{16}{Nennen Sie drei anormale Ausbreitungsmöglichkeiten im 70 cm-Band oder 2m Band.}
\question{17}{Welche Betriebsverfahren werden im Satellitenfunkverkehr angewendet?}
\question{18}{Was verstehen Sie unter ,,Scatter-Verbindung''?}
\question{19}{Was verstehen Sie unter ,,EME-Verbindung''?}
\question{20}{Was verstehen Sie unter ,,Meteor-Scatter''?}
\question{21}{Was verstehen Sie unter ,,Tropo-Scatter''?}
\question{22}{Was verstehen Sie unter Überreichweiten, was unter dem Funkhorizont?}
\question{23}{Wodurch werden starke Überreichweiten im 70 cm-Band verursacht?}
\question{24}{Wie verhalten Sie sich bei Überreichweitenbedingungen, wenn Sie im Relaisbetrieb arbeiten?}
\question{25}{Wie können Sie sich über die herrschenden Ausbreitungsbedingungen informieren?}
\question{26}{Welche Faktoren beeinflussen die erzielbare Reichweite im 2m-Band?}
\question{27}{Erklären Sie die Bedeutung der auch im Sprechfunk verwendeten Q-Gruppen: QSO - QSY - QRL.}
\question{28}{Erklären Sie die Bedeutung der auch im Sprechfunk verwendeten Q-Gruppen: QRM - QRB - QSB.}
\question{29}{Erklären Sie die Bedeutung der auch im Sprechfunk verwendeten Q-Gruppen: QRT - QSL.}
\question{30}{Erklären Sie die Bedeutung der im Sprechfunk verwendeten Abkürzungen \\ 73- 55- 88- CL.}
\question{31}{Was versteht man unter der Betriebsart ,,Packet-Radio'', welche Betriebsverfahren werden dabei angewendet?}
\question{32}{Welche Faktoren beeinflussen die erzielbare Reichweite im 70cm-Band?}
\question{33}{Was verstehen Sie unter ,,Split-Betrieb''?}
\question{34}{Welche Verfahren werden bei ATV-Betrieb im 70 cm-Band angewendet und welche Besonderheiten sind dabei zu beachten?}
\question{35}{Wie gehen Sie bei der Planung einer Amateurfunkverbindung zu einem bestimmten Ort vor?}
\question{36}{Wie teilen Sie der Gegenstation den Standort ihrer Amateurfunkstelle mit?}
\question{37}{Was ist hinsichtlich der Herstellung oder Veränderung von Geräten für den Amateurfunkverkehr im 2m oder 70 cm-Band zu beachten?}
\question{38}{Sie haben einen abstimmbaren Leistungsverstärker - wie stimmen Sie ihn ab?}

\def\chap{Technische Grundlagen \class{1}}

\question{01}{Ohmsches und Kirchhoff’sches Gesetz}
\question{02}{Begriff Leiter, Halbleiter, Nichtleiter}
\question{03}{Kondensator, Begriff Kapazität, Einheiten - Verhalten bei Gleich- und Wechselspannung}
\question{04}{Spule, Begriff Induktivität, Einheiten - Verhalten bei Gleich- und Wechselspannung}
\question{05}{Wärmeverhalten von elektrischen Bauelementen}
\question{06}{Stromquellen (Kenngrössen)}
\question{07}{Sinus- und nicht-sinusförmige Signale}
\question{08}{Was verstehen Sie unter dem Begriff Skin-Effekt?}
\question{09}{Gleich- und Wechselspannung - Kenngrößen}
\question{10}{Was verstehen Sie unter dem Begriff Permeabilität?}
\question{11}{Serien- und Parallelschaltung von $R$, $L$, $C$}
\question{12}{Was verstehen Sie unter dem Begriff Dielektrikum?}
\question{13}{Wirk-, Blind- und Scheinleistung bei Wechselstrom.}
\question{14}{Begriff elektrischer Widerstand (Schein-, Wirk- und Blindwiderstand), Leitwert}
\question{15}{Berechnen Sie den induktiven Blindwiderstand einer Spule mit $30~\mu H$ bei $7$ MHz (Werte sind variabel)}
\question{16}{Berechnen Sie den kapazitiven Blindwiderstand eines Kondensators von 500 pF bei 10 MHz (Werte sind variabel)}
\question{17}{Der Transformator - Prinzip und Anwendung}
\question{18}{Der Resonanzschwingkreis - Kenngrößen}
\question{19}{Der Resonanzschwingkreis - Anwendungen in der Funktechnik}
\question{20}{Berechnen Sie die Resonanzfrequenz eines Schwingkreises mit folgenden Werten: L = 15 H, C = 30 pF (Werte sind variabel)}
\question{21}{Filter – Arten, Aufbau, Verwendung und Wirkungsweise}
\question{22}{Was sind Halbleiter?}
\question{23}{Die Diode - Aufbau, Wirkungsweise und Anwendung}
\question{24}{Der Transistor - Aufbau, Wirkungsweise und Anwendung}
\question{25}{Die Elektronenröhre - Aufbau, Wirkungsweise und Anwendung}
\question{26}{Arten von Gleichrichterschaltungen - Wirkungsweise}
\question{27}{Stabilisatorschaltungen}
\question{28}{Hochspannungsnetzteil - Aufbau, Dimensionierung und Schutzmaßnahmen}
\question{29}{Welche Arten von digitalen Bauteilen kennen Sie? - Wirkungsweise}
\question{30}{Was sind elektronische Gatter? - Wirkungsweise}
\question{31}{Messung von Spannung und Strom am Beispiel eines vorgegebenen Stromkreises}
\question{32}{Erklären Sie die prinzipielle Wirkungsweise eines Griddipmeters, Anwendung und Funktion}
\question{33}{Erklären Sie die Funktionsweise eines HF-Wattmeters}
\question{34}{Erklären Sie die Funktionsweise eines Oszillografen (Oszilloskop)}
\question{35}{Erklären Sie die Funktionsweise eines Spektrumanalysators}
\question{36}{Begriff Demodulation}
\question{37}{Zeichnen Sie das Blockschaltbild eines Überlagerungsempfängers}
\question{38}{Was verstehen Sie unter Spiegelfrequenz und Zwischenfrequenz?}
\question{39}{Erklären Sie die Kenngrößen eines Empfängers - Empfindlichkeit, intermodulationsfreier Bereich, Eigenrauschen}
\question{40}{Erklären Sie den Begriff des Rauschens. - Auswirkungen auf den Empfang.}
\question{41}{Mischer in Empfängern - Funktionsweise und mögliche technische Probleme}
\question{42}{Nichtlineare Verzerrungen - Ursachen und Auswirkungen}
\question{43}{Empfängerstörstrahlung - Ursachen und Auswirkungen}
\question{44}{Mikrofonarten - Wirkungsweise}
\question{45}{Prinzip, Arten und Kenngrößen der Einseitenbandmodulation}
\question{46}{Prinzip, Arten und Kenngrößen der Pulsmodulation}
\question{47}{Erklären Sie die wichtigsten Anwendungen der digitalen Modulationsverfahren}
\question{48}{Erklären Sie die Begriffe CRC und FEC}
\question{49}{Prinzip und Kenngrößen der Frequenzmodulation}
\question{50}{Prinzip und Kenngrößen der Amplitudenmodulation}
\question{51}{Erklären Sie den Begriff Modulation (analoge und digitale Verfahren)}
\question{52}{Oszillatoren - Grundprinzip, Arten}
\question{53}{Erklären Sie den Begriff VCO}
\question{54}{Erklären Sie den Begriff PLL}
\question{55}{Erklären Sie den Begriff DSP}
\question{56}{Erklären Sie die Begriffe sampling, anti aliasing filter, ADC/DAC}
\question{57}{Merkmale, Komponenten, Baugruppen eines Senders}
\question{58}{Zweck von Puffer- und Vervielfacherstufen, Aufbau}
\question{59}{Aufbau einer Senderendstufe, Leistungsauskopplung}
\question{60}{Anpassung eines Senderausganges an eine symmetrische oder asymmetrische Antennenspeiseleitung}
\question{61}{Der Antennentuner, Wirkungsweise, 2 typische Beispiele}
\question{62}{Antennenzuleitungen - Aufbau, Kenngrößen}
\question{63}{Erklären Sie den Begriff Balun. Aufbau, Verwendung und Wirkungsweise}
\question{64}{Der Dipol - Aufbau, Kenngrößen und Eigenschaften}
\question{65}{Die Vertikalantenne - Aufbau, Kenngrößen und Eigenschaften}
\question{66}{Gekoppelte Antennen - Aufbau, Kenngrößen und Eigenschaften}
\question{67}{Strahlungsdiagramm einer Antenne}
\question{68}{Die Yagi-Antenne - Aufbau, Kenngrößen und Eigenschaften}
\question{69}{Breitbandantennen - Aufbau, Kenngrößen und Eigenschaften}
\question{70}{Die Parabolantenne - Aufbau, Kenngrößen und Eigenschaften}
\question{71}{Erklären Sie den Begriff Wellenwiderstand}
\question{72}{Stehwellen und Wanderwellen, Ursachen und Auswirkungen}
\question{73}{Strahlungsfeld einer Antenne, Gefahren}
\question{74}{Aufbau und Kenngrößen eines Koaxialkabels}
\question{75}{Erklären Sie den Begriff Dezibel am Beispiel der Anwendung in der Antennentechnik}
\question{76}{Was versteht man unter Richtantennen, Anwendungsmöglichkeiten}
\question{77}{Welche Kenngrößen von Antennen kennen Sie und wie können sie gemessen werden?}
\question{78}{Dimensionieren Sie einen Halbwellendipol für f = 3.6 MHz ; V = 0.97 (Werte sind variabel)}
\question{79}{Bestimmen Sie die effektive Strahlungsleistung bei folgenden Gegebenheiten: Senderleistung: 200 Watt; Dämpfung der Antennenleitung: 6 dB/100m; Kabellänge : 50 m; Gewinn: 10 dB (Werte sind variabel)}
\question{80}{Bestimmen Sie die effektive Strahlungsleistung bei folgenden Gegebenheiten: Senderleistung 100 Watt; Dämpfung der Antennenleitung 12 dB/100m; Kabellänge 25 m; Rundstrahlantenne mit Gesamtwirkungsgrad von 50 \% (Werte sind variabel)}
\question{81}{Langdrahtantennen - Aufbau, Kenngrößen und Eigenschaften}
\question{82}{Zweck von Radials / Erdnetz bei Vertikalantennen - Dimensionierung}
\question{83}{Blitzschutz für Antennenanlagen}
\question{84}{Sicherheitsabstände bei Antennen}
\question{85}{Erklären Sie den Begriff ,,elektromagnetisches Feld''. Kenngrößen?}
\question{86}{Begriff elektrisches und magnetisches Feld; Abschirmmaßnahmen für das elektrische bzw. das magnetische Feld?}
\question{87}{Erklären Sie den Begriff ,,EMV'' und dessen Bedeutung im Amateurfunk}
\question{88}{Erklären Sie den Begriff ,,EMVU'' und dessen Bedeutung im Amateurfunk}
\question{89}{Erklären Sie den Begriff ,,Trap'', Aufbau und Wirkungsweise}
\question{90}{Was versteht man unter einem Hohlraumresonator, Anwendung.}
\question{91}{Funkentstörmaßnahmen im Bereich Stromversorgung der Amateurfunkstelle}
\question{92}{Funkentstörmaßnahmen bei Beeinflussung durch hochfrequente Ströme und Felder}
\question{93}{Was sind Tastklicks, wie werden sie vermieden?}
\question{94}{Erklären Sie die Begriffe: ,,Unerwünschte Aussendungen'', ,,Ausserbandaussendungen'', ,,Nebenaussendungen'' (spurious emissions)}
\question{95}{Erklären Sie den Begriff: ,,Splatter'' - Ursachen und Auswirkungen}
\question{96}{Erklären sie den Begriff ,,schädliche Störungen''}
\question{97}{Prinzipieller Aufbau einer Relaisfunkstelle und einer Bakenfunkstelle}
\question{98}{Definieren Sie den Begriff ,,Senderleistung''}
\question{99}{Definieren Sie den Begriff ,,Spitzenleistung''}
\question{100}{Definieren Sie den Begriff ,,belegte Bandbreite''}
\question{101}{Definieren Sie den Begriff ,,Interferenz in elektronischen Anlagen''; beschreiben Sie Ursachen und Gegenmassnahmen}
\question{102}{Erklären Sie die Begriffe ,,Blocking'', ,,Intermodulation''}
\question{103}{Welche Gefahren bestehen für Personen durch den elektrischen Strom?}
\question{104}{Was ist beim Betrieb von Hochspannung führenden Geräten zu beachten?}
\question{105}{Definieren Sie die Gefahren durch Gewitter für die Funkstation und das Bedienpersonal, beschreiben Sie Vorbeugemassnahmen}

\def\chap{Technische Grundlagen \class{3,4}}

\question{01}{In welchem Zusammenhang stehen die Größen Strom – Spannung - Widerstand in einem Stromkreis?}
\question{02}{Was versteht man unter einem Kurzschluß - wie entsteht er?}
\question{03}{Nennen Sie Stromquellen}
\question{04}{Kenngrößen einer Gleichstromquelle. Kenngrößen einer Wechselstromquelle - Gefahrengrenze?}
\question{06}{Nennen Sie die wichtigsten Eigenschaften von Ohm'schen Widerständen, Induktivitäten und Kapazitäten.}
\question{07}{Was verstehen Sie unter dem Begriff ,,Fehlanpassung''?}
\question{08}{Was verstehen Sie unter dem Begriff ,,Transformation''?}
\question{09}{Prinzipieller Aufbau eines Kommunikationssystems. Erläutern Sie die Wirkungsweise von Mikrophon und Lautsprecher bzw. Kopfhörer.}
\question{11}{Prinzipieller Aufbau eines Senders}
\question{12}{Funktionsprinzip des Oszillators}
\question{13}{Prinzipieller Aufbau eines Empfängers}
\question{14}{Prinzip des Überlagerungsempfängers. Was verstehen Sie unter dem Begriff Zwischenfrequenz?}
\question{16}{Was verstehen Sie unter dem Begriff Modulation?}
\question{17}{Kenngrößen der Amplitudenmodulation}
\question{18}{Kenngrößen der Frequenzmodulation}
\question{19}{Definieren Sie den Begriff ,,belegte Bandbreite''. Arten und Vorteile der Einseitenbandmodulation?}
\question{21}{Begriff Dezibel (Werte fragen: zB 3 dB, 6 dB, 10 dB, 30 dB Leistungssteigerung)}
\question{22}{Was ist eine Diode - Wirkungsweise, Verwendung?}
\question{23}{Was ist ein Transistor - Wirkungsweise, Verwendung?}
\question{24}{Was versteht man unter ,,AGC'' und ,,AFC''? Erklären Sie die Empfängerkenngrößen - Empfindlichkeit, Eigenrauschen, Empfangsmischprodukte}
\question{26}{Was versteht man unter dem S/N - Verhältnis?}
\question{27}{Erklären Sie die Begriffe ,,digital'' und ,,analog''.}
\question{28}{Was versteht man unter der Ausgangsleistung, was unter der Verlustleistung?}
\question{29}{Was versteht man unter der Strahlungsleistung? (Beispiel vorgeben, zB. Sender mit 10 W Ausgangsleistung; Antennenkabel mit 3 dB Dämpfung; Antenne mit 10 dB Gewinn)}
\question{30}{Begriff Speiseleitung (Antennenzuleitung) - Kenngrößen?}
\question{31}{Auswirkung(en) des Stehwellenverhältnisses (SWR)?}
\question{32}{Kenngrößen einer Antenne am Beispiel des Dipols}
\question{33}{Vertikalantenne - Eigenschaften}
\question{34}{Die Yagi-Antenne - Aufbau, Eigenschaften, Kenngrößen}
\question{35}{Dipolkombinationen (Zeilen, Spalten)}
\question{36}{Die Parabolantenne - Aufbau, Eigenschaften, Kenngrößen}
\question{37}{Mobilantennen - Aufbau, Eigenschaften, Kenngrößen, Montageort}
\question{38}{Grundausrüstung einer Amateurfunkstelle für Sprechfunk (Komponenten)}
\question{39}{Grundausrüstung einer Amateurfunkstelle für Packet Radio}
\question{40}{Grundausrüstung einer Amateurfunkstelle für ATV-Betrieb}
\question{41}{Was versteht man unter Betriebserde- was unter Blitzschutzerde?}
\question{42}{Was versteht man unter BCI, TVI?}
\question{43}{Maßnahmen gegen BCI, TVI?}
\question{44}{Was versteht man unter dem ``SQUELCH'' - wozu dient er?}
\question{45}{Wie bestimmt man die Resonanzfrequenz einer Antenne?}
\question{46}{Was ist ein SWR-Meter, wo und wie wird es eingesetzt?}
\question{47}{Was versteht man unter einem ``Antennen-Tuner''?}
\question{48}{Was versteht man unter ``Dopplershift''?}
\question{49}{Komponenten einer Amateurfunkstation für Satellitenfunk}
\question{50}{Abstrahlung und Ausbreitung elektromagnetischer Wellen, Feldstärke?}
\question{51}{Was versteht man unter Freiraumausbreitung?}
\question{52}{Welche Einflüsse haben Hindernisse auf die UKW-Ausbreitung?}
\question{53}{Definieren Sie den Begriff ,,Schädliche Störung''?}
\question{54}{Definieren Sie den Begriff ,,Senderleistung''?}
\question{55}{Definieren Sie den Begriff ,,Spitzenleistung''?}
\question{56}{Definieren Sie den Begriff ,,unerwünschte Aussendung''?}

\end{document}
